
\section*{Part 2}

There is a correspondance between the sorting tree of the \emph{RandQS} algorithm
and the random treaps.
This correspondance is basically that the choice of pivots in RandQS
corresponds to the choice of priorities in a random treap.
Both the RandQS tree and the treap sorts the keys from left to right.
Which pivot is chosen, then, is simply the one with highest priority,
since must have higher priority than all nodes in its subtrees.
Similar to the lecture slides let $x_i$ denote the key with the
$i$th lowest priority.
Let $p_i$ denotes the priorities.
And let $X_{ij} = 1_{\{x_i \text{ is an ancester of } x_i\}}$.
Suppose the $x_k$ has priority $-\infty$.
Then $ X_{ik} = 1 \iff p_i = \max\{p_i, \dots, p_{k-1} \} $.
So $\bb{E}(X_{ik}) = P(X_{ik} = 1) = \frac{1}{k - 1 + 1 - i} = \frac{1}{k - i}$.
So the expected depth of $x_k$ is
$ \bb{E}(\sum_{i<k} X_{ik}) = \sum_{i=1}^{k-1} \frac{1}{d} = H_{k-1} $.

When deleting a key $x$ in a random treap we assign $x$ priority $-\infty$.
Then make rotations until $x$ is a leaf.
The number of rotations needed is thus exactly the difference in depth between
$x$ and a supposed key with priority $-\infty$.
