\section*{Part 1}
\subsection*{Problem 1}
Analogously to the proof of theorem 6,
we wish to find a bound on the probability that more than $\frac{3}{4}2^{\ell}$ 
keys from $S \setminus \left\{ q \right\}$ hash to some given $\ell$-intervali, $I$. \\ \\
As in the proof let $X_x$ indicate whether $h(x) \in I$, where $x \in S \setminus \left\{ q \right\}$. 
Then $X=\sum_{x \in S \setminus \left\{ q \right\}}^{}X_x$ counts the elements different from $q$ that hash to $I$. Then
\begin{align*}
	&\mu=EX=\sum_{x \in S \setminus \left\{ q \right\}}^{}EX_x \\
	=&\sum_{x \in S \setminus \left\{ q \right\}} P(h'(x)  \in I) \\
	=&\sum_{x \in S \setminus \left\{ q \right\}} \sum_{y \in I}^{}P\left( h'(x) =y \right) \\
	<&\sum_{x \in S \setminus \left\{ q \right\}}2^{\ell}\left( 1/t+1/p \right) \\
	<& n 2^{\ell}\left( 1/t+1/p \right) \\
	\leq & n2^{\ell} \left( \frac{1}{t}+\frac{1}{24t} \right) \\
	\leq & n 2^{\ell} \frac{25}{24t} \\
	\leq & \frac{2}{3}2^{\ell} \frac{25}{24} = \frac{50}{72}2^{\ell}
\end{align*}
Hence we get
\begin{align*}
	X \geq \frac{3}{4}2^{\ell} \implies X-\mu \geq \frac{3}{4}2^{\ell}-\mu \geq \frac{1}{18}2^{\ell} >\frac{1}{15}\sqrt{2^{\ell}\mu}
\end{align*}
and using the fourth moment bound we obtain
\begin{align*}
	P\left( X \geq \frac{3}{4}2^{\ell} \right) \leq \frac{4}{\left( \frac{1}{15}2^{\ell}\right)^4} 
	= \frac{4 \cdot 15^4}{2^{2\ell}} \in \mathcal{O}\left( \frac{1}{2^{2 \ell}} \right)
\end{align*}
Hence the conclusion in the proof of theorem 6 still holds.

\subsection*{Problem 2}
By lemma 2 we get that if there is a run of length $r \geq 2^{\ell+2}$ 
hence for a given run R, the probability that this run is longer than $2^{\ell+2}$ is bounded by $4 P_{\ell}$. \\ 
Throughout the rest of the argument we will take $\ell=\lg(C \lg n)$. 
Then $2^{\ell+2}=4C \lg n \in \mathcal{O}(n)$. \\ \\
Now let $I$ be some $\ell$-interval. Then as in the proof of theorem 6 let $X_x$ indicate the event
$\left( h(x) \in I \right)$ for any $x \in \left[ n \right]$ and $X=\sum_{x \in \left[ n \right]}^{}X_x$. \\
Then $X$ counts the number of keys hashing to $I$ and we still have $\mu=EX \leq \frac{2}{3}2^{\ell}$, 
and because we are assuming full randomness the $X_x$ are independent poisson trials. \\
Hence, using Chernoff bounds and the upper bound for $EX$ we get 
\begin{align*}
	&P(X \geq \frac{3}{4}2^{\ell}) \\
	\leq& P\left( X > (1+\frac{1}{7})\frac{2}{3}2^{\ell} \right) \\
	\leq & \exp \left( \frac{-\frac{2}{3}2^{\ell}\frac{1}{49}}{4} \right) \\
	=& \exp \left( C \lg n \frac{2}{12 \cdot 49} \right) \\
	=& n^{-C\frac{1}{294}} \leq \frac{1}{4n^{11}}
\end{align*}
for appropriately chosen $C$. 
Observe that we can use this version of the Chernoff bound because $\frac{1}{7} \leq 2e-1$. \\ \\
Now by the observation derived from lemma 2 we have
\begin{align*}
	P\left( |R| \geq 2^{\ell+2} \right) \leq 4 P_{\ell} \leq \frac{1}{n^{11}}
\end{align*}
and hence 
\begin{align*}
	P\left( \text{some run is longer than } 2^{\ell+2} \right) 
	\leq \sum_{R}^{}P\left( |R| \geq 2^{\ell+2} \right)
	\leq n \frac{1}{n^{11}}=\frac{1}{n^{10}}
\end{align*}
since there can be at most $n$ runs, one for each key. \\ \\
Hence by taking complements we get
\begin{align*}
	P\left( \text{all runs are no longer than } 2^{\ell+2} \right) \geq 1-\frac{1}{n^{10}}
\end{align*}
as desired.
\subsection*{Problem 3}
Using the same notation as in the proof of theorem 6 we observe that
\begin{align*}
	VX_x=\left( 1-\frac{2^{\ell}}{t} \right)\frac{2^{\ell}}{t} \leq \frac{2^{\ell}}{t}
\end{align*}
and since the hash function is 3-independent the $X_x$'s will be pairwise independent so
\begin{align*}
	VX=\sum_{x \in S \setminus \left\{ q \right\}}^{}VX_x \leq n \frac{2^{\ell}}{t} \leq \frac{2}{3}2^{\ell}
\end{align*}
and similar to the proof of theorem 6 we obtain
\begin{align*}
	X \geq \frac{3}{4}2^{\ell} \implies X-\mu \geq \frac{1}{12}2^{\ell} > \frac{1}{10} \sqrt{2^{\ell} VX}
\end{align*}
yielding (by Chebyshev)
\begin{align*}
	P\left( X \geq \frac{3}{4}2^{\ell} \right) \leq 
	P\left( |X-\mu| > \frac{1}{10}2^{\ell/2} \sqrt{VX} \right) \leq \frac{100}{2^{\ell}} \in \mathcal{O}\left( \frac{1}{2^{\ell}} \right)
\end{align*}
So the expected operation time is
\begin{align*}
	\mathcal{O}\left( 1+\sum_{\ell=0}^{\lg t}2^{\ell}P_{\ell} \right)= 
	\mathcal{O}\left( 1+\sum_{\ell=0}^{\lg t} 1 \right) =
	\mathcal{O} \left( 1+ \lg t \right)= \mathcal{O}\left( \lg n \right)
\end{align*}
as desired.
