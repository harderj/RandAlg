\section{Problem 7.4}

Suppose two multisets
$X=\langle x_1, \dots, x_n\rangle,Y=\langle y_1, \dots, y_n \rangle$ are given.
We can associate to a multiset (e.g. X) the polynomial
\[ p_X(z) = (z - x_1)(z - x_2) \dots (z - x_n) \]
It is easy to see that identical multisets give rise to the same polynomial,
(by commutativity of multiplication).
On the other hand if $X$ and $Y$ are not identical,
the multiplicity of at least one root must differ
(say non-roots have multiplicity zero) so the polynomials will differ.
We can now envoke the Schwartz-Zippel theorem\ldots %TODO

