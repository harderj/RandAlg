\section{Problem 7.4}

Suppose two integer multisets
$X=\langle x_1, \dots, x_n\rangle,Y=\langle y_1, \dots, y_n \rangle$ are given.
We can associate to a multiset (e.g. X) the polynomial
\[ p_X(z) = (z - x_1)(z - x_2) \dots (z - x_n) \]
It is easy to see that identical multisets give rise to the same polynomial,
(by commutativity of multiplication).
On the other hand if $X$ and $Y$ are not identical,
the multiplicity of at least one root must differ
(say non-roots have multiplicity zero) so the polynomials will differ.
This means we can tell apart the multisets by determining if their polynomials differ.
Observe that $p_X$ is a polynomial in 1 variable which has degree $n$.
By the Schwartz-Zippel theorem we can choose a field $\bb{F}$
containing $x_1, y_1, \dots, x_n, y_n$
and a subset $S \subseteq \bb{F}$ with $|S|=2n$ 
such that
\[ P(p_X(r) - p_Y(r) = 0 \mid p_X - p_Y \not \equiv 0) \leq \frac{1}{2} \]
when $r \in S$ is chosen uniformly at random.
If we assume multiplication in $\bb{F}$ is $O(1)$
we get a Monte Carlos algorithm with $2^{-m}$ probability of failure
running in $O(m n)$ time
- an improvement to the $O(n \log n)$ of the sorting algorithm.

We now move on to discuss what happens if we have to consider
the sizes of the integers involved and
the time complexity of comparison and multiplication of such.
Assume the integers in the multisets are all less than $k$.
We can convert them to $\ell = \ceil{\log_2 (k) }$ bit strings in $O(n)$ time.
To determine equality of two of the integers takes $O(\ell)$ time. 
Thus the sorting algorithm requires $O(n \log n \log k)$ time.

Looking at our randomized algorithm we have a problem with the values of the
polynomials becoming very large,
in fact at the order $k^n$ requiring us to work with $\ell n$ bit numbers
if we use $\R$ as the field.
However we could choose a prime $2k>p>k$ and work with the finite field $\bb{F}_p$.
This garantees $\ell = \ceil{\log_2 (p)} = O(\log(k))$ bit numbers are enough.
Multiplication of $\ell$ bit numbers is actually $O(\ell^2) = O(\log k)$.
So we can evaluate our polynomials in $O(n \log k)$ time.
All in all we get $O(n m \log k)$ running time
with success probability $1 - 2^{-m}$ still an improvement to $O(n \log n \log k)$.

However if we want to convert from Monte Carlo to Las Vegas:
To garantee that two multisets are equal we need to check at least $n+1$
different values from $S$ yielding a bad running time of
$O(n^2 \log k)$.
