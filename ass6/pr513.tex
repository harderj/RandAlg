\section*{Problem 5.13}
Let $v_1,\cdots,v_n$ denote the vertices of the graph. \\
As in section 5.6 of Motwani \& Raghavan we consider the decision tree of the experiment. \\
Each node is labelled by two sets $[A,B]$ corresponding to the assignments made so far,
i.e. the root $r$ is labelled $[\emptyset, \emptyset]$ 
and if $a$ is any node at level $i$ labelled by $[A,B]$ we have $\# (A \cup B)=i$,
and if $c$ and $d$ are its left and right child respectively, they will be labelled
$[A \cup \left\{ v_{i+1} \right\}, B]$ and $[A, B \cup \left\{ v_{i+1} \right\}]$ respectively,
and the algorithm proceeds equiprobably to either child. \\ \\
Now for a node, $a$, in the decision tree let $E(a)$ denote the expected number of crossing edges
conditioned on reaching node $a$. Then clearly 
\begin{align*}
	E(a)=\frac{E(c)+E(d)}{2}
\end{align*}
so that $\max\left\{ E(c), E(d) \right\} \geq E(a)$.
Furthermore we have seen in theorem 5.1 that $E(r) \geq m/2$. \\
Hence it is possible to make choices that do not decrease the expected number of crossing edges
all the way from the root to a leaf, 
and since there is no more randomness at a leaf, $l$, the number of crossing edges is equal to $E(l)$,
and hence making the choices such that the expected number of crossing edges do not decrease
will yield a partition where this number is at least $m/2$. \\ \\
So we are left with the task of determining which of $E(c)$ and $E(d)$ is the larger. \\
To this end consider a node, $a$, at level $i$ in the tree, 
i.e. proceeding to either the left or the right child corresponds to assigning
$v_{i+1}$ to $A$ or $B$ respectively. Let $k_A$ denote the number of crossing edges from
$v_{i+1}$ to $A$ and define $k_B$ analogously. \\ \\
Now observe that after assigning $v_{i+1}$
we still have for any edge $(u,v)$, where either or both of it end points have not yet been assigned,
that it will become a crossing edge with probability $\frac{1}{2}$.
Furthermore, the number of such edges will be the same regardless of the assignment of $v_{i+1}$; 
it will be the number of such edges before the assignment of $v_{i+1}$
minus the number of edges one of whose end points is $v_{i+1}$ and the other is in $A \cup B$.
Call this number $N_0$. \\
Then since assigning $v_{i+1}$ to $A$ adds $k_B$ crossing edges and assigning it to $B$
adds $k_A$ crossing edges we obtain that
$E(c)=K_0+k_B+\frac{N_0}{2}$ and $E(d)=K_0+k_A+\frac{N_0}{2}$, 
where $K_0$ is the number of crossing edges from $A$ to $B$ before assigning $v_{i+1}$.
Hence we have reduced the question to determining the larger of $k_A$ and $k_B$
which can be done by just checking the other end point of all of $v_{i+1}$'s incident edges. \\ \\
If there are no multiple edges there can be at most $n-1$ of these and
so this procedure is polynomial when performed on all $n$ vertices. \\
If we allow multiple edges we can merge any multiple edges between to vertices into one
giving it an integer weight corresponding to the number of edges between the two vertices.
This weight is then used in computing $k_A$ and $k_B$ 
and these computations are essentially equivalent to the case of no multiple edges. \\ \\
Thus by the above considerations the deterministic algorithm of computing $k_A$ and $k_B$ 
for each vertex in turn and assigning it to $B$ if $k_A$ is the largest and vice versa
will obtain a partition with at least $m/2$ crossing edges in polynomial time as desired.
