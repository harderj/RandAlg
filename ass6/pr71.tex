\section*{Problem 7.1}
As observed in the proof of theorem 7.1, the task is equivalent to
computing $P(Mr=0 \ | \ M \neq 0)$. \\
So assume that $M$ is a non-zero $n \times n$ matrix and sample $r=(r_1,\cdots,r_n)^T$
by sampling each entry independently and uniformly from $\left\{ 0,1 \right\}$. \\ \\
Now since $M$ is non-zero, at least one of its row, $M_1,\cdots,M_n$ is non-zero.
Let $M_i$ denote this row. \\
Then with $x=(x_1,\cdots,x_n)^T$, $M_ix$ is a multivariate (non-zero) polynomial of total degree $1$,
and hence by Schwartz-Zippel $P\left( M_ir=0 \right) \leq \frac{1}{2}$,
since the $r_1,\cdots,r_n$ are sampled independently and uniformly from a subset of $\R$ of size $2$. \\ \\
Hence 
\begin{align*}
	P\left( Mr=0 \right)=P\left( M_1r=0,\cdots,M_nr=0 \right) \leq P\left( M_ir=0 \right) \leq \frac{1}{2}
\end{align*}
proving theorem $7.1$ using the Schwartz-Zippel theorem.
