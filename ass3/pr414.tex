\section*{Problem 4.14}

Let us look at the sorting tree of some execution of the \emph{RandQS} algorithm on some set $S$.
Say that at some node the algorithm chooses a \emph{good pivot} provided that it selects an element from its given subset $S' \subset S$ that is results in two subsets (elements of higher resp. lower order) with size no larger than $3/4 |S'|$.
We claim that the chance of this selection by the algorithm is $\geq 1/2$.
To see this note that the number of elements in the \emph{lower} subset $L$ follows a uniform distribution on $\{0,\cdots,n-1\}$, where $n = |S'|$.
The number of elements in the \emph{higher} subset $H$ depends entirely on $|L|$ in that $|L| + |H| = n-1$.
The chance that a bad pivot is chosen is therefore
\begin{align*}
  P(|L|>3/4 n \lor |H|>3/4 n) &= 2 P(|L|>3/4 n)
  \\ &= 2 P(|L| \geq \ceil{3/4 n})
  \\ &= 2 \left(\frac{n-1 - \ceil{3/4 n} + 1}{n} \right)
  \\ &\leq 2 \frac{1}{4} = 1/2
\end{align*}
Thus the chance of a good pivot chosen is $\geq 1/2$.
Fix an element $x \in S$ and let $h$ denote the length of the path from the root of the tree to $x$.
For each step on this path if a good pivot is chosen the number of remaining elements in the batch containing $x$ is reduced by at least $3/4$.
Solving
\[(3/4)^k n \leq 1 \iff k + \log_{3/4}(n) \leq 0 \iff k \geq \log_{4/3}(n) \]
we see that if more than $m = \log_{4/3}(n) = \ln(n) / \ln(4/3) $ good pivots is chosen the height of $x$ has been attained.
Now we calculate the chance that less than $m$ good pivots are chosen in $8 m$ steps.
The choices of pivots are independent and each is good with probability at least $1/2$.
By the lower Chernoff bound with $\delta = 3/4$ and $\mu = 1/2 m$ the probability of having too few good pivots is less than 
\[ e^{-\frac{m}{2} \left(\frac{3}{4}\right)^2 / 2} \leq e^{-\frac{m}{8}} = e^{-\frac{\ln(n)}{\ln(4/3)}} = n^{-\frac{1}{\ln(4/3)}} \leq n^{-3} \] 
Thus for any $x \in S$ the probability that $x$ has height greater than $\frac{8}{\ln(4/3)} \ln(n) = h_m$ is less than $n^{-3}$.
By union bounding the chance that the whole tree has height more than $h_m$ is less than $n^{-2}$. 
The sum of the heights of all elements equal the number of comparisons of the algorithm.
Therefore the number of comparisons is at most $\frac{8}{\ln(4/3)} n \ln(n)$ with probability $1 - n^{-2}$ proving that \emph{RandQS} is $O(n \ln(n))$ with high probability. 
