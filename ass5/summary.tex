\section*{The probabilistic method}
Two main ideas:
\begin{itemize}
	\item Any random variable assumes at least one value no less and one value no greater
		than its expectation with positive probability
	\item If an object chosen according to some distribution from a universe satisfies a property with positive probability
		then at least one object from this universe satisifies said property.
\end{itemize}
Usefull observations for existence proofs: 
construct thought experiments and show that the output of such experiments satisfy certain properties with positive probability
or proof statements about the expected outcome of the experiment.
\subsection*{Example}
For an undirected graph $G\left( E,V \right)$ with $n$ vertices and $m$ edges,
there is a partition of the vertices into sets $A$ and $B$ such that
\begin{align*}
	\# \left\{ (u,v) \in E \ | \ u \in A \ \text{and} \ v \in B \right\} \geq \frac{m}{2}
\end{align*}
To see this consider the experiment of choosing $A$ and $B$ 
by independently and equiprobably assigning each vertex to either $A$ or $B$
and compute the expected size of the above set. \\ \\
\textbf{Theorem} \indent For $n$ large enough there exists a bipartite graph $G(L,R,E)$
with $\#L=n$ and $\#R = 2^{\log^2 n}$ such that
\begin{itemize}
	\item Every subset of $n/2$ vertices of $L$ has at least $2^{\log^2 n}-n$ neighbors in $R$
	\item No vertex of $R$ has more thatn $12 \log^2n$ neighbors
\end{itemize}
Can be proven by considering the experiment of letting each vertex in $L$ choose $2^{\log^2n}4 \log^2n/n$ neighbors in $R$
independently at random and considering the probability of the thus obtained graph satisfying the above requirements 
(its strictly positive). \\ \\
Such expanding graphs can be used for \textit{probability amplification}:
in the case where we have an RP algorithm, $A$, for deciding membership of a language $L$
with witnesses from $\Z_n$ we have seen how two point sampling can be used to obtain an error probability of
less than $1/t$ with $t$ trials, using $2 \log n$ random bits to sample two random numbers from $\Z_n$. \\
Using $\log^2n$ bits of randomness we can distinguish a vertex of $R$ in such a graph
and consider its (at most $12 \log^2n$) neighbors in $L$, $r_1,\cdots,r_k$
using these as potential witnesses. \\
By the above theorem the error probability obtained by using these neighbors in stead of random numbers is at most
$n/n^{\log n}$. \\ \\
We observe that the above theorem only asserts the existence of a graph with the stated properties
and does not construct it, but at least the theorem guarantees that a search for such a graph will not be in vain,
and once one such graph is found it can be used for probability amplification over and over again.
\newpage
