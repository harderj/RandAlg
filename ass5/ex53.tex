\section*{Problem 5.3}
\subsection*{(a)}

Each vertice survives the deletion process with probability $\frac{1}{d}$
so the number of surviving vertices is binomially distributed 
with trial parameter $n$ and success parameter $\frac{1}{d}$ and hence the expected number of surviving vertices
is $\frac{n}{d}$. \\ \\
Now observe that an edge is deleted if and only if one of its end points is. 
Hence it survives if and only if both its end points do. 
Since the deletions are performed independently, this happens with probability $\frac{1}{d^2}$,
so the number of surviving edges is binomially distributed
with trial parameter $\frac{nd}{2}$ and success parameter $\frac{1}{d^2}$ and as before we expect
$\frac{nd}{2}\cdot \frac{1}{d^2}=\frac{n}{2d}$ surviving edges.

\subsection*{(b)}
Consider the following extention of the experiment: \\
After the deletion process iterate through the remaining edges
deleting one of its end points (in any fashion - deterministically or probabilistically)
and its incident edges. \\ \\
This will yield an independent subset (which might be empty!) 
since any vertice still remaining after this procedure will have all its neighbors deleted. \\ \\
If $v_0$ and $e_0$ denote the number of vertices resp. edges remaining after the first deletion round
then the resulting independent subset after the second deletion process will have at least
$\max\left\{ 0, v_0-e_0 \right\}$ vertices. \\ \\
By subproblem $(a)$ the expected number of vertices in the independent set resulting from this experiment
run on a graph with $n$ vertices and $\frac{nd}{2}$ edges is at least
$\max\left\{ 0, \frac{n}{d}-\frac{n}{2d} \right\}=\max\left\{ 0, \frac{n}{2d} \right\}=\frac{n}{2d}$. \\
Since a random variable assumes at least one value no less than its expectation with positive probability
this yields the existence of at least one independent set of at least $\frac{n}{2d}$ vertices.
